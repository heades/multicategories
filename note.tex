\documentclass{article}
\usepackage{parskip}
\usepackage{amsmath,amssymb,amsthm}
\usepackage{mathpartir}
\usepackage{diagrams}
\usepackage{stmaryrd}

% Theorems
\newtheorem{theorem}{Theorem}
\newtheorem{lemma}[theorem]{Lemma}
\newtheorem{fact}[theorem]{Fact}
\newtheorem{corollary}[theorem]{Corollary}
\newtheorem{definition}[theorem]{Definition}
\newtheorem{remark}[theorem]{Remark}
\newtheorem{proposition}[theorem]{Proposition}
\newtheorem{notn}[theorem]{Notation}
\newtheorem{observation}[theorem]{Observation}

% Commands that are useful for writing about type theory and programming language design.
%% \newcommand{\case}[4]{\text{case}\ #1\ \text{of}\ #2\text{.}#3\text{,}#2\text{.}#4}
\newcommand{\interp}[1]{\llbracket #1 \rrbracket}
\newcommand{\normto}[0]{\rightsquigarrow^{!}}
\newcommand{\join}[0]{\downarrow}
\newcommand{\redto}[0]{\rightsquigarrow}
\newcommand{\nat}[0]{\mathbb{N}}
\newcommand{\fun}[2]{\lambda #1.#2}
\newcommand{\CRI}[0]{\text{CR-Norm}}
\newcommand{\CRII}[0]{\text{CR-Pres}}
\newcommand{\CRIII}[0]{\text{CR-Prog}}
\newcommand{\subexp}[0]{\sqsubseteq}
%% Must include \usepackage{mathrsfs} for this to work.
\newcommand{\powerset}[0]{\mathscr{P}}


\title{A Note on Multicategories}
\author{Harley Eades III\footnote{email: harley.eades@gmail.com}\\ Georgia Regents University Augusta}
\date{}

\begin{document}

\maketitle  

In categorical logic it is customary to model sequents of the form
$A_1,\ldots,A_n \vdash B$ in a monoidal category as morphism $f :
\interp{A_1} \otimes \cdots \otimes \interp{A_n} \to \interp{B}$.  For
example, modeling propositional intuitionistic logic requires the
monoidal category to be at least cartesian, and $\otimes$ to be the
cartesian product, while modeling a sequent in propositional linear
logic requires only tensor.  This works well, but what if the logic
one is seeking to model categorically does not have a suitable notion
of a tensor product that can model the left-hand side of sequents?
Then the notion of a multicategory just might be the structure
suitable for such a model.

In an ordinary category morphisms have by definition only a single
source object and a single target object.  Multicategories generalize
morphisms to have multiple source objects denoted $f : X_1,\ldots,X_n
\to Y$.

\nocite{*}
\bibliographystyle{plain}
\bibliography{references}

\end{document}

